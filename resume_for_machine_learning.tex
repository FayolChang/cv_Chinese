%!TEX program = xelatex
%%%%%%%%%%%%%%%%%%%%%%%%%%%%%%%%%%%%%%%%%
% Friggeri Resume/CV
% XeLaTeX Template
% Version 1.0 (5/5/13)
%
% This template has been downloaded from:
% http://www.LaTeXTemplates.com
%
% Original author:
% Adrien Friggeri (adrien@friggeri.net)
% https://github.com/afriggeri/CV
%
% License:
% CC BY-NC-SA 3.0 (http://creativecommons.org/licenses/by-nc-sa/3.0/)
%
% Important notes:
% This template needs to be compiled with XeLaTeX and the bibliography, if used,
% needs to be compiled with biber rather than bibtex.
%
%%%%%%%%%%%%%%%%%%%%%%%%%%%%%%%%%%%%%%%%%

\documentclass[print]{friggeri-cv} % Add 'print' as an option into the square bracket to remove colors from this template for printing

\usepackage{ctex}




\addbibresource{bibliography.bib} % Specify the bibliography file to include publications

\begin{document}

\header{张}{发幼}{机器学习相关职位} % Your name and current job title/field

%----------------------------------------------------------------------------------------
%	SIDEBAR SECTION
%----------------------------------------------------------------------------------------

\begin{aside} % In the aside, each new line forces a line break
\section{联系方式}
曹安公路 4800,
嘉定区, 上海
201804
~
+86 189 1827 8022
~
\href{mailto:zhangfayou@gmail.com}{zhangfayou@gmail.com}
%\href{http://www.smith.com}{http://www.smith.com}
\href{http://fayolchang.github.io/hexoblog}{github}
% \section{languages}
% english mother tongue
% spanish \& italian fluency
\section{编程语言}
{\color{red} $\varheartsuit$} R
Python, Matlab Mathematica
% \section{standardized tests}
% \textbf{TOEFL:} 94  \textbf{GRE:} Verbal 153 Math 162 AW 3
% \section{interests}
% \textbf{professional:}
% data analysis
% programming
% \textbf{personal:}
% bridge, running
% \section{courses}
% {2012-2014}
% {computer science related}
% {TongJi University}
% {Modern Optimization In Management Science
% Numerical Optimization
% Convex Optimization
% Management Information System
% Multivariate Statistics}
\end{aside}

%----------------------------------------------------------------------------------------
%	EDUCATION SECTION
%----------------------------------------------------------------------------------------

\section{教育背景}

\begin{entrylist}
%------------------------------------------------
\entry
{2012-2015}
{ {\normalfont 管理科学与工程}硕士}
{同济大学}
{GPA:86.7/100\\
\emph{基于 DEA 与随机森林的上市公司财务困境预测} \\
本论文通过引入 DEA 效率作为一个预测变量,运用递归特征选择方法选择对预测影响最大的特征,然后使用支持向量机、朴素贝叶斯、神经网络、随机森林等机器学习方法,来预测上市公司2年后是否会被特别处理。研究显示 DEA 效率对预测结果有显著影响,本文预测精度比通常研究提高5-10个百分点.}
%------------------------------------------------
\entry
{2003-2007}
{{\normalfont 材料科学与工程}学士}
{哈尔滨工业大学}
{负责数据处理\\
参与水泥混凝土抗冻融耐久性能研究}
%------------------------------------------------
\end{entrylist}




%%-------------------------------------------------------------------------------
%  skills section
%%------------------------------------------------------------------------------
\section{所学技能}
\begin{entrylist}

\entry
{}
{熟练}
{}
{\emph{算法}\\
基本数据结构与算法:\\
quick sort,Binary Search Tree,Dijkstra,Floyd等\\
机器学习相关算法:\\
线搜索, 梯度下降, 共轭梯度, 拟牛顿法, 模拟退火,  K近邻,支持向量机,boosting,bagging等
\emph{软件}\\%
 Microsoft Office,latex,git, sed \& awk}



\entry
{}
{熟悉}
{}
{\emph{算法}\\
BFGS, 遗传算法, 禁忌搜索, 随机森林, 神经网络等\\ %
\emph{软件 \& 操作系统}\\
SQL, linux, Mac OS X,基础 bash 命令
}

\end{entrylist}




%----------------------------------------------------------------------------------------
%	WORK EXPERIENCE SECTION
%----------------------------------------------------------------------------------------

\section{经历经验}

\begin{entrylist}
%------------------------------------------------
\entry
{2014}
{华为杯全国研究生数学建模竞赛}
{同济大学}
{二等奖 \\
负责模型构建和编程实现 \\
模型考虑了使用不同轿运车的成本、装卸成本、里程成本、多个运输目的地;
构建了整数规划模型,使得总费用最小\\
编程使用启发式算法和分支界定法解决这个整数规划模型
}
%-----------------------------------------------------

% \entry
% {2012-2014}
% {Research Assistant, \normalfont{Supply Chain \& Industrial Engineering Lab}}
% {TongJi University}
% {Data cleaning, Data filtering and Data transformation for machine learning algorithms\\
%  Investigated feature selection via Random Forest and Rough set\\
%  Designed a new feature for better prediction accuracy\\
%  Studied various machine learning algorithms for better prediction\\
%  Applied machine learning algorithms to predict business failure
% }

%-----------------------------------------------------


\entry
{2007-2011}
{江苏博特新材料有限公司}
{江苏南京}
{浙江省销售与技术服务\\
参与沪杭高铁项目
负责杭长高速公路项目}

%-----------------------------------------------------

% \entry
% {2007}
% {Building Materials Lab}
% {Harbin Institute of Technology}
% {\emph{Freeze-Thaw Resistance and Durability of Cement Concrete}\\
% Developed strong skills in C\# and SQL\\
% }

%\emph{1\textsuperscript{st} Year Analyst} \\
% Developed spreadsheets for risk analysis on exotic derivatives on a wide array of commodities (ags, oils, precious and base metals), managed blotter and secondary trades on structured notes, liaised with Middle Office, Sales and Structuring for bookkeeping. \\
% Detailed achievements:
% \begin{itemize}
% \item Learned how to make amazing coffee
% \item Finally determined the reason for \textsc{PC LOAD LETTER}:
% 	\begin{itemize}
% 		\item Paper jam
% 		\item Software issues:
% 		\begin{itemize}
% 			\item Word not sending the correct data to printer
% 			\item Windows trying to print in letter format
% 		\end{itemize}
% 		\item Coffee spilled inside printer
% 	\end{itemize}
% \item Broke the office record for number of kitten pictures in cubicle
% \end{itemize}

%------------------------------------------------
% \entry
% {2010--2011}
% {LEHMAN BROTHERS}
% {London, United Kingdom}
% {\emph{Summer Intern} \\
% Rated "truly distinctive" for Analytical Skills and Teamwork.}
%------------------------------------------------
% \entry
% {2008-2009}
% {Buy More}
% {Burbank, California}
% {\emph{Computer Repair Specialist} \\
% Worked in the Nerd Herd and helped to solve computer problems by asking customers to turn their computers off and on again.}
%------------------------------------------------
\end{entrylist}















%----------------------------------------------------------------------------------------
%	COMMUNICATION SKILLS SECTION
%----------------------------------------------------------------------------------------

\section{所获证书}

\begin{entrylist}
%------------------------------------------------
\entry
{2014}
{数据科学}
{约翰霍普金斯大学 on Coursera}
{数据科学家工具箱\\
 R语言编程\\
数据获取与清理\\
探索性数据分析\\
可重复性研究\\
统计推断\\
回归模型\\
机器学习实战\\
数据产品开发}
%{Presented the research I conducted for my Masters of Commerce degree.}
%------------------------------------------------
% \entry
% {2010}
% {Poster}
% {Annual Business Conference, Oregon}
% {As part of the course work for BUS320, I created a poster analyzing several local businesses and presented this at a conference.}
%------------------------------------------------
\end{entrylist}












%----------------------------------------------------------------------------------------
%	INTERESTS SECTION
%----------------------------------------------------------------------------------------
\section{所学课程}
\begin{entrylist}
\entry
{2012-2014}
{}
{同济大学}
{管理优化方法,
数值最优化,
凸优化, 
管理信息系统,
高级运筹学,
高级应用统计学,
计量经济学,
高级管理学,
物流与供应链管理
}
\end{entrylist}




%----------------------------------------------------------------------------------------
%   AWARDS SECTION
%----------------------------------------------------------------------------------------
% \section{所获奖项}
% \begin{entrylist}
% %------------------------------------------------
% \entry
% {2012-2014}
% {研究生奖学金}
% {同济大学,经济与管理学院}
% {}
% %------------------------------------------------
% \end{entrylist}


\section{语言成绩}
% \begin{entrylist}
% \entry
% {}
% {}
% {}
{\textbf{TOEFL:} 96  \textbf{GRE:} Verbal 153 Math 162 AW 3 六级:通过}
% \end{entrylist}



\section{兴趣爱好}
% \begin{entrylist}
% \entry
% {}
% {}
% {}
{\textbf{professional:} 编程,数据分析
 \textbf{personal:} 游泳,跑步}
% \end{entrylist}

%----------------------------------------------------------------------------------------
%	PUBLICATIONS SECTION
%----------------------------------------------------------------------------------------

\section{发表论文}

%\begin{enumerate}

[1]马卫民,李彬,徐博,张发幼. 考虑节点中断和需求波动的可靠供应链网络设计问题[J]. 系统工程理论与实践,,:1.

[2]段永瑞,张发幼. 基于随机森林和 modified-SBM 的上市公司财务危机预警[J].经营管理者,(5)2015 

%\end{enumerate}
% \printbibsection{article}{} % Print all articles from the bibliography
%
% \printbibsection{book}{books} % Print all books from the bibliography
%
% \begin{refsection} % This is a custom heading for those references marked as "inproceedings" but not containing "keyword=france"
% \nocite{*}
% \printbibliography[sorting=chronological, type=inproceedings, title={international peer-reviewed conferences/proceedings}, notkeyword={france}, heading=subbibliography]
% \end{refsection}
%
% \begin{refsection} % This is a custom heading for those references marked as "inproceedings" and containing "keyword=france"
% \nocite{*}
% \printbibliography[sorting=chronological, type=inproceedings, title={local peer-reviewed conferences/proceedings}, keyword={france}, heading=subbibliography]
% \end{refsection}
%
% \printbibsection{misc}{other publications} % Print all miscellaneous entries from the bibliography
%
% \printbibsection{report}{research reports} % Print all research reports from the bibliography

%----------------------------------------------------------------------------------------
\end{document}
